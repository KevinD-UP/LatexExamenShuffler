
\documentclass[a4paper,10pt,french]{article}

\usepackage[T1]{fontenc}
\usepackage[utf8]{inputenc}
\usepackage{babel}
\usepackage{pdfsync}

\newcommand{\nterm}[1]{\textcolor{red}{#1}}
\newcommand{\term}[1]{\textcolor{blue}{#1}}
\newcommand{\bterm}[1]{\mathbin{\textcolor{blue}{#1}}}
\newcommand{\LL}{\mathrm{LL}}
\newcommand{\LR}{\mathrm{LR}}
\newcommand{\follow}{\mathrm{FOLLOW}}
\newcommand{\first}{\mathrm{FIRST}}
\newcommand{\eps}{\mathrm{EPS}}
\def\mybox#1#2{\fbox{\parbox[b][#1][b]{#2}{~}}}

\newtheorem{exi}{}
\newenvironment{exo}{\begin{exi}\em}{\end{exi}}

\begin{document}
\pagestyle{empty}
\hrule
\medbreak\medbreak
\centerline{\textsf\large{\bf Grammaire et analyse: TD noté du vendredi 14 avril 2023}}
\medbreak

\noindent
Les réponses seront données uniquement à l'intérieur des cadres.
Dans tous les exercices qui suivent, les lettres majuscules sont les non-terminaux, les autres symboles sont les 
terminaux.

\begin{exo}

  Soit la grammaire d'axiome~$S$ suivante:
\begin{itemize}
\item[] $S \rightarrow aSa \mid TU$
\item[] $T \rightarrow bbT\mid b $
\item[] $U \rightarrow cc\mid ccc $
\end{itemize}


\begin{enumerate}
\item Donner une dérivation pour le mot $aabbbccaa$.
  
\mybox{2cm}{\linewidth}

\item Parmi chacun des mots suivants, indiquer avec {\bf Oui} ou {\bf Non} s'il appartient au langage engendré par la grammaire.

\begin{center}
\begin{tabular}{|p{16mm}|p{16mm}|p{16mm}|p{16mm}|p{16mm}|p{16mm}|p{16mm}|}
\hline
\hfil $\varepsilon$	&\hfil $accca$	&\hfil $abcca$	&\hfil $aabcccaa$	&\hfil $abbbca$	&\hfil $aaabccaa$	&\hfil $bbbccc$ \\
\hline
%n&n&o&o&n&n&o\\
&&&&&&\\
\hline
\end{tabular}
\end{center}
\item
  Décrire le langage engendré par cette grammaire (par exemple, avec la notation $\{\ldots \mid \ldots \}$, comme pour $\{a^nb^n \mid n \geq 0\}$).
  
 \mybox{1.5cm}{\linewidth}

 
\end{enumerate}

 \end{exo}



\begin{exo}

  Soit la grammaire suivante où l'axiome est~$T$, l'ensemble des terminaux est $\{ a, c, (, )\}$ et l'ensemble des non-terminaux $\{T, U, V\}$:

\begin{itemize}
\item[] $T \rightarrow a(U)$
\item[] $U \rightarrow VU\mid \varepsilon$
\item[] $V \rightarrow c\mid T$
\end{itemize}
  \begin{enumerate}

  \item
    Donner l'ensemble $\eps$ des non-terminaux effaçables (annulables).
   \mybox{1cm}{5cm}
  
  \item
    Donner l'ensemble $\first_1$ de chacun des non-terminaux avec l'algorithme du cours. 

    \begin{tabular}{|lp{1cm}p{3cm}p{3cm}p{3cm}|}
      \hline
      &&&&\\
      Graphe && T &U &V\\
      &&&&\\
      Initialisation &&&&\\
      &&&&\\
     Propagation&&&&\\
     &&&&\\
      \hline
    \end{tabular}

    \vspace{0.5cm}

  \item
    Donner l'ensemble $\follow_1$ de chacun des non-terminaux avec l'algorithme du cours.

    \begin{tabular}{|lp{1cm}p{3cm}p{3cm}p{3cm}|}
      \hline
      &&&&\\
      Graphe && T &U &V\\
      &&&&\\
      Initialisation &&&&\\
      &&&&\\
      Propagation&&&&\\
      &&&&\\
     \hline
    \end{tabular}

    \vspace{0.5cm}

  \item Est-ce que la grammaire est $\LL(1)$? Justifier.
  
  \mybox{5cm}{\linewidth}
  \end{enumerate}
  \end{exo}

\begin{exo}
On considère la grammaire d'axiome~$S$ suivante:
\begin{itemize}
\item[] $S \rightarrow T\$$
\item[] $T \rightarrow aTb \mid Ud$
\item[] $U \rightarrow a$
\end{itemize}

\begin{enumerate}
\item Cette grammaire n'est pas $\LR(0)$. Donner le début de la construction
de l'automate caractéristique qui le montre.

\mybox{4cm}{\linewidth}

\item Donner l'automate caractéristique déterministe $\LR(1)$ pour cette
grammaire. Est-elle $\LR(1)$?

\mybox{9cm}{\linewidth}

\end{enumerate}

\end{exo}

\end{document}




  
  
