\documentclass[a4paper,10pt,french]{article}

\usepackage[T1]{fontenc}
\usepackage[utf8]{inputenc}
\usepackage{babel}
\usepackage{pdfsync}

\newcommand{\nterm}[1]{\textcolor{red}{#1}}
\newcommand{\term}[1]{\textcolor{blue}{#1}}
\newcommand{\bterm}[1]{\mathbin{\textcolor{blue}{#1}}}
\newcommand{\LL}{\mathrm{LL}}
\newcommand{\LR}{\mathrm{LR}}
\newcommand{\follow}{\mathrm{FOLLOW}}
\newcommand{\first}{\mathrm{FIRST}}
\newcommand{\eps}{\mathrm{EPS}}
\def\mybox#1#2{\fbox{\parbox[b][#1][b]{#2}{~}}}

\newtheorem{exi}{}
\newenvironment{exo}{\begin{exi}\em}{\end{exi}}

\begin{document}
\pagestyle{empty}
\hrule
\medbreak\medbreak
\centerline{\textsf\large{\bf Grammaire et analyse: TD noté du vendredi 14 avril 2023}}
\medbreak

\noindent
Donnez vos réponses uniquement à l'intérieur des cadres.
Dans tous les exercices qui suivent, les minuscules sont les 
terminaux, les majuscules sont les non-terminaux.

\begin{exo}

  Soit la grammaire suivante où $T$ est l'axiome, l'ensemble des terminaux est $\{ a, c, (, )\}$ et l'ensemble des non-terminaux $\{T, U, V\}$:

\begin{itemize}
\item[] $T \rightarrow a(U)$
\item[]   $U \rightarrow VU\mid \varepsilon$
\item[] $V \rightarrow c\mid T$
\end{itemize}
  \begin{enumerate}

  \item
    Donnez l'ensemble $EPS$ des non-terminaux effaçables.
   \mybox{1cm}{5cm}
  
  \item
    Donnez les ensembles $FIRST_1$ pour chacun des non-terminaux avec l'algorithme du cours. 

    \begin{tabular}{|lp{3cm}p{3cm}p{3cm}|}
      \hline
      &&&\\
      Graphe & T &U &V\\
      &&&\\
      Initialisation &&&\\
      &&&\\
     Propagation&&&\\
     \hline
    \end{tabular}

    \vspace{0.5cm}

  \item
    Donnez les ensembles $FOLLOW_1$ pour chacun des non-terminaux   avec l'algorithme du cours.

    \begin{tabular}{|lp{3cm}p{3cm}p{3cm}|}
      \hline
      &&&\\
      Graphe & T &U &V\\
      &&&\\
      Initialisation &&&\\
      &&&\\
     Propagation&&&\\
     \hline
    \end{tabular}

    \vspace{0.5cm}

  \item Est-ce que la grammaire est $LL(1)$? Justifiez votre réponse.
  
  \mybox{5cm}{\linewidth}
  \end{enumerate}
  \end{exo}

\begin{exo}
On considère la grammaire suivante:
\begin{itemize}
\item[] $S \rightarrow T\$$
\item[] $T \rightarrow aTb \mid c \mid Ud$
\item[] $U \rightarrow a$
\end{itemize}

\begin{enumerate}
\item Cette grammaire n'est pas LR(0). Donnez le début de la construction
de l'automate caractéristique qui le montre.

\mybox{4cm}{\linewidth}

\item Donnez l'automate caractéristique dét. LR(1) pour cette
grammaire. Elle est LR(1) ?

\mybox{9cm}{\linewidth}

\end{enumerate}

\end{exo}

\begin{exo}

  Soit la grammaire suivante dont $S$ est l'axiome:
\begin{itemize}
\item[] $S \rightarrow aSa \mid TU$
\item[]   $T \rightarrow bbT\mid \varepsilon $
\item[]   $U \rightarrow c\mid \varepsilon $
\end{itemize}


\begin{enumerate}
\item Donnez une dérivation pour le mot $aabbcaa$.
  
\mybox{1cm}{\linewidth}

\item Parmi les mots suivants, indiquez avec {\bf Oui} ou {\bf Non}, s'ils sont
dans le langage de la grammaire.

\begin{tabular}{|c|c|c|c|c|c|}
\hline
\hspace*{2ex}$\epsilon$\hspace*{2ex}&$aa$&$aaa$&$aacaa$&$abba$&$bbcc$\\
\hline
&&&&&\\
\hline
\end{tabular}

\item
  Décrivez le langage généré par cette grammaire. (Par exemple, avec $\{\ldots \mid\ldots \}$)
  
 \mybox{1.5cm}{\linewidth}

 
\end{enumerate}

 \end{exo}


\end{document}




  
  
