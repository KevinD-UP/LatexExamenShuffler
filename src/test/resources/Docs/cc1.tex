\documentclass[a4paper,10pt,french]{article}

\usepackage{babel}

\usepackage[utf8]{inputenc}
\usepackage[T1]{fontenc}
\usepackage{amsfonts}
\usepackage{amsmath}
\usepackage{amssymb}
\usepackage{stmaryrd}

% format standard, centré
 \setlength{\oddsidemargin}{2mm}
 \setlength{\evensidemargin}{2mm}
 \setlength{\textwidth}{155mm}
 \setlength{\textheight}{245mm}
 \setlength{\topmargin}{0mm}
% 1 mm entre chaque paragraphe
 \setlength{\parskip}{1mm}

\newtheorem{exi}{}
\newenvironment{exo}{\begin{exi}\em}{\end{exi}}

\newcommand{\lsem}{\llbracket}
\newcommand{\rsem}{\rrbracket}

\newcommand{\inte}[2]{\llbracket #1 \rsem {#2}}

\begin{document}

\vspace*{-1cm}
\hrule
\medbreak
\centerline{\textsf{O{\small utils} et L{\small ogique} (OL4)}}
\medbreak
\hrule
\medbreak
\centerline{\textsf{\phantom{(} ~~~~~~~CC1~~~~~~~\phantom{)}}}
\medbreak
\hrule

\bigskip
\bigskip

{\bf NOM:} \hspace*{4cm} {\bf PRENOM:} \hspace*{3cm} {\bf Numéro
  d'étudiant:}\\

Utilisez uniquement les espaces encadrées prévues pour répondre.

\begin{exo}

  On considère les affectations suivantes:$
v_1 = [ x\mapsto 1] \quad
v_2= [ x\mapsto 1, y\mapsto 1 ]  \quad
v_3= [ y\mapsto 1, z\mapsto 1 ]$.
Remplissez le tableau suivant ($0$ ou $1$ dans les trois premières colonnes et $vrai$ ou $\mathit{faux}$ dans le reste)~:

\begin{tabular}{|c|c|c|c|c|c|c|}
  \hline
    Formule $p$ &$\inte{p}{v_1}$&$\inte{p}{v_2}$&$\inte{p}{v_3}$&$p$ valide ?&$p$ satisfaisable ?&$p$ contradictoire ?\\[0.5ex]
    \hline
    $((\neg x \vee y) \land x)$&&&&&&\\[0.5ex]
    \hline
    $(\neg y \land (z \land (x \land y)))$&&&&&&\\[0.5ex]
    \hline
    $((x \land z) \vee ((\neg x \land z) \vee \neg z))$&&&&&&\\[0.5ex]
    \hline
  \end{tabular}
\end{exo}

\begin{exo}
  Formalisez les phrases suivantes en logique propositionnelle
  (vous pouvez utiliser l'implication).
  Indiquez pour chaque variable propositionnelle que vous utilisez
  à quoi elle correspond.

  \begin{enumerate}
  \item Paris est belle.
  \item Si Paris est belle, alors Berlin est belle.
  \item Si Berlin n'est pas belle, alors ni Paris ni Rome sont belles.
  \end{enumerate}


  \framebox[\textwidth]{
      \rule[-0.4cm]{0mm}{4cm}
  }
é
\end{exo}


\begin{exo}
  On considère l'affectation $v =  []$, qui associe à chaque variable la valeur
  de vérité $0$.
  Montrer par induction structurelle sur les formules que pour toute formule
  $p$ qui ne contient pas $\neg$, on a $\inte{p}{v}=0$.
  N.B.~: dans la preuve le cas $\neg p$ n'est pas considéré.
  
  \framebox[\textwidth]{
    \rule[-0.4cm]{0mm}{8cm}
}
\end{exo}

\begin{exo}
  On considère la fonction $taille(p)$ qui étant donnée une formule propositionnelle $p$
  désigne sa taille (en nombre de n\oe{u}ds, y compris les feuilles) de
  l'arbre syntaxique correspondant. Par exemple $taille((x \vee (\neg x \land y)))=6$ et
  $taille(\neg x) = 2$.
  
  On considère également la fonction $hauteur(p)$ qui étant donnée une formule
  propositionnelle $p$ désigne la hauteur de l'arbre syntaxique correspondant.
  Par exemple, $hauteur(((\neg x \land y) \vee x))=4$ et
  $hauteur((x\vee(\neg x \land y)))=4$ et
  $hauteur(\neg x) = 2$.
  Complétez les définitions récursives suivantes:
  \begin{itemize}
  \item $taille(x) = $ \framebox[1cm]{\rule[-0.2cm]{0mm}{0.5cm}},
    $hauteur(x) = $ \framebox[1cm]{\rule[-0.2cm]{0mm}{0.5cm}}
  \item $taille(\neg p) = $ \framebox[3cm]{\rule[-0.2cm]{0mm}{0.5cm}},
    $hauteur(\neg p) = $ \framebox[3cm]{\rule[-0.2cm]{0mm}{0.5cm}}
  \item $taille((p \land q)) = $ \framebox[4cm]{\rule[-0.2cm]{0mm}{0.5cm}},
    $hauteur((p \land q)) = $ \framebox[4cm]{\rule[-0.2cm]{0mm}{0.5cm}}
    \item $taille((p \vee q)) = $ \framebox[4cm]{\rule[-0.2cm]{0mm}{0.5cm}},
        $hauteur((p \vee q)) = $ \framebox[4cm]{\rule[-0.2cm]{0mm}{0.5cm}}
  \end{itemize}
  
  Montrez par induction structurelle que pour toute formule propositionnelle
  $p$ on a $taille(p) \geq hauteur(p)$.\\[2ex]
   \framebox[\textwidth]{
    \rule[-0.4cm]{0mm}{13cm}
   }
\end{exo}

\end{document}
